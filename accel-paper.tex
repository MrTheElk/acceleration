\documentclass[twocolumn,showpacs,preprintnumbers,nofootinbib,prd,
superscriptaddress,10pt]{revtex4-1}

\usepackage{amsmath,amssymb}
\usepackage[normalem]{ulem}
\usepackage{textcomp}
\usepackage{hyperref}
\usepackage{bm}
\usepackage{graphicx}
\usepackage{psfrag}
\usepackage[usenames,dvipsnames]{xcolor}
\usepackage[utf8]{inputenc}

\graphicspath{{./figures/}}
% 
\allowdisplaybreaks[4]

\newcommand{\sub}[1]{_{\text{#1}}}
\newcommand{\super}[1]{^{\text{#1}}}
\newcommand{\uvec}[1]{\bm{\hat{#1}}}
\newcommand{\dvec}[1]{\dot{\bm{#1}}}
\newcommand{\ddvec}[1]{\ddot{\bm{#1}}}
\newcommand{\dddvec}[1]{\dddot{\bm{#1}}}
\newcommand{\duvec}[1]{\dot{\bm{\hat{#1}}}}
\newcommand{\dduvec}[1]{\ddot{\bm{\hat{#1}}}}
\newcommand{\ddduvec}[1]{\dddot{\bm{\hat{#1}}}}
\newcommand{\ord}[1]{\mathcal{O} \left( #1 \right)}

\newcommand{\ak}[1]{{\textcolor{Red}{\sf{[AK: #1]}} }}

\begin{document}

\title{Peculiar acceleration effects on stellar-origin gravitational wave binaries}

\date{\today}

\author{Nicola Tamanini}
\affiliation{Max-Planck-Institut für Gravitationsphysik, Albert-Einstein-Institut, Am Mühlenberg 1,
14476 Potsdam-Golm, Germany}
\affiliation{Laboratoire Astroparticule et Cosmologie, CNRS UMR 7164, Université Paris-Diderot, 10
rue Alice Domon et Léonie Duquet, 75013 Paris, France}

\author{Antoine Klein}
\affiliation{CNRS, UMR 7095, Institut d'Astrophysique de Paris, 98 bis Bd Arago, 75014 Paris, 
France}

\author{Enrico Barausse}
\affiliation{CNRS, UMR 7095, Institut d'Astrophysique de Paris, 98 bis Bd Arago, 75014 Paris, 
France}

\author{Chiara Caprini}
\affiliation{Laboratoire Astroparticule et Cosmologie, CNRS UMR 7164, Université Paris-Diderot, 10
rue Alice Domon et Léonie Duquet, 75013 Paris, France}

\author{Camille Bonvin}
\affiliation{D\'{e}partment de Physique Th\'{e}orique, Universit\'{e} de Gen\`{e}ve,
24 quai Ernest-Ansermet, 1211 Gen\`{e}ve 4, Switzerland}



\begin{abstract}
We look at peculiar acceleration.
\end{abstract}


\pacs{
 04.30.-w, %Gravitational waves
 04.30.Tv %Gravitational-wave astrophysics
}

\maketitle


\section{Introduction}


\section{Theory}


\subsection{Peculiar acceleration effects on gravitational waveforms}

The gravitational radiation emitted by a binary system is affected by the redshift
of the source in the detector frame. The GW observed by the detector takes on the following form in the 
source frame:
\begin{align}
 h(t) &= \frac{G M \nu}{d_C c^2} \sum_{n \geq 0} A_n e^{-i n \phi} + c.c., \\
  \dot{\phi} &= \frac{c^3}{G M} y^3,
\end{align}
where $M$ is the intrinsic total mass of the system, $d_C$ is the comoving distance 
from the source to the detector, $\phi$ is its orbital phase, and $y = (G M \omega / c^3)^{1/3}$ is a post-Newtonian parameter used to describe the system. In the presence of a redshift between the source and the detector, the phase evolution equation is affected as
\begin{align}
 \dot{\phi}\sub{obs} &= \frac{1}{1+z} \frac{c^3}{G M} y^3.
\end{align}

The effect of the redshift can be absorbed into the total mass, which becomes the \emph{redshifted mass} $M_z = (1+z) M$. The amplitude of the GW has then to be 
corrected accordingly. One way of doing this is to use the luminosity distance
$d_L = (1+z) d_C$. The GW in the detector frame thus becomes
\begin{align}
 h(t) &= \frac{G M_z \nu}{d_L c^2} \sum_{n \geq 0} A_n e^{-i n \phi} + c.c., \\
  \dot{\phi} &= \frac{c^3}{G M_z} y^3.
\end{align}

In the presence of a peculiar acceleration, this picture changes as the redshift becomes
a function of time.






\subsection{Peculiar accelerations from astrophysical origin}


\section{Measurement errors}




\section{Astrophysical populations}





\section{Results}





\section{Conclusion}




\acknowledgments


A.~K. and E.~B. are supported by H2020-MSCA-RISE-2015 Grant No. StronGrHEP-690904.
This work was 
supported by the Centre National d'{\'E}tudes Spatiales.



\bibliography{accel-paper}

\end{document}
